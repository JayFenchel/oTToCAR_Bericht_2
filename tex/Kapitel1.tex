%%%%%%%%%%%%%%%%%%%%%%%%%%%%%%%%%%%%%%%%%%%%%%
% Author: 	Hannes Heinemann
% Date:			31.03.2014
% Content:	Kapitel1.tex
%%%%%%%%%%%%%%%%%%%%%%%%%%%%%%%%%%%%%%%%%%%%%%
\chapter{Kinematisches Modell} \label{chap:kinematik}
In diesem Kapitel werden kinematische Zusammenhänge dargestellt, die dem Verständnis des dynamischen Modells dienen. Zunächst wird das Fahrzeug im globalen Koordinatensystem lokalisiert. Mithilfe der kinematischen Bedingungen an die Räder für das vorliegende Fahrzeug wird die Verwendung des Einspurmodells begründet. Dieses Modell ist Grundlage des Kapitels \ref{chap:dynamic}. Auf eine Herleitung eines erweiterten kinematischen Modells für niedrige Geschwindigkeiten, wie in \cite{Werling2010}, wird verzichtet und stattdessen ein dynamisches Modell auch für kleine Geschwindigkeiten verwendet.

%%%%%%%%%%%%%%%%%%%%%%%%%%%%%%%%%%%%%%%%%%%%%%
\section{Weltkoordinatensystem} \label{sec:Weltkoord}
Es wird in der gesamten Arbeit davon ausgegangen, dass sich das Modellfahrzeug auf einer horizontalen Ebene befindet.
Diese Annahme stimmt mit den im Carolo-Cup gestellten Rahmenbedingungen überein. 
Wie die Position aus der Umgebungswahrnehmung des Fahrzeuges in Roboterkoordinaten übersetzt werden kann, wird im Folgenden erläutert.