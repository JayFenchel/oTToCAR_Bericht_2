%%%%%%%%%%%%%%%%%%%%%%%%%%%%%%%%%%%%%%%%%%%%%%
% Author: 	Viktoria Wiedmeyer
% Date: 	18.09.2013
% File:	intro.tex
% Content: 	Einleitung
%%%%%%%%%%%%%%%%%%%%%%%%%%%%%%%%%%%%%%%%%%%%%%
\chapter{Einleitung}
Im VEGDFDIULS wurde bereits im Herbst 2012 ein Gesetz erlassen, das den Einsatz autonomer Fahrzeuge im Straßenverkehr erlaubt, solange sich ein Fahrer am Steuer befindet (vgl. \cite{Biermann2012}). 

Im August 2013 fand die erste ca. 100km lange "`autonome Langstreckenfahrt im Überland- und Stadtverkehr mit seriennaher Sensorik"' mit dem "`Forschungsfahrzeug S 500 INTELLIGENT DRIVE"' \cite{Daimler2013} von Mercedes-Benz in Deutschland statt. 

Die öffentliche Diskussion wird nicht nur durch ähnliche Ereignisse und die Verfügbarkeit von Fahrerassistenzsystemen wie Spurhalteassistenten und Einparkautomatik für den privaten Verbraucher bestimmt. Im Jahr 2012 waren $\SI{85}{\percent}$ der Unfälle in Deutschland auf das Fehlverhalten von Kraftfahrzeugführern zurückzuführen (vgl. \cite{destatis2013}). Die Idee eines autonomen Fahrzeugs dient vor allem dazu die Sicherheit im Straßenverkehr zu erhöhen und die Anzahl von Unfällen zu senken. 

In der vorliegenden Arbeit werden theoretische Überlegungen dargestellt, die im Rahmen der Vorbereitung zum Hochschulwettbewerb "`Carolo-Cup"' der TU Braunschweig benötigt werden.
Das interdisziplinäre Team oTToCAR der Otto"=von"=Guericke"=Universität entwickelt ein autonomes Modellfahrzeug für die Teilnahme am Wettbewerb.
Im Wettbewerb müssen verschiedene Fahraufgaben möglichst schnell und fehlerfrei absolviert werden. Dabei handelt es sich um das Abfahren eines Rundkurses unter verschiedenen Bedingungen und um das Einparken. Außerdem müssen Konzepte für die Fahrzeugherstellung und die Bewältigung der Szenarien präsentiert werden (vgl. \cite{carolo2014}).

Zur Entwicklung des Fahrzeugs hat das oTToCAR-Team verschiedene Aufgaben definiert. Dabei handelt es sich u.a. um die Konstruktion des Modellautos inklusive Chassis, Motortreiber, Sensorik usw. sowie die Inbetriebnahme der On-Board Kommunikation bis hin zum Einsatz echtzeitfähiger Regelungskonzepte um schließlich die Wettbewerbsszenarien realisieren zu können.
In der vorliegenden Arbeit werden die kinematische Bedingungen betrachtet, um ein Verständnis für das Fahrzeugverhalten zu gewinnen (Kapitel \ref{chap:kinematik}). Durch dynamische Modellierung (Kapitel \ref{chap:dynamic}) wird das Fahrzeugverhalten weiter abgebildet und schließlich simuliert (Kapitel \ref{chap:Simulation}). Zum Abgleich mit dem realen Modellauto werden Konzepte der Parameterschätzung anhand von Messdaten (Kapitel \ref{chap:ParamEstim}) vorgestellt.



%Motivation autonomes Fahrzeug
%-Bedeutung
%-Historie/aktueller Stand
%-Vorteile
%-Unfälle
%
%%Öffentlichkeit RoboCup, Lange Nacht der Wissenschaft (Sensorik)
%
%Der Hochschulwettbewerb „Carolo-Cup“ bietet Studententeams die Möglichkeit, sich mit der Entwicklung und Umsetzung von autonomen Modellfahrzeugen auseinander zu setzen. Die Herausforderung liegt in der Realisierung einer bestmöglichen Fahrzeugführung in unterschiedlichen Szenarien, die sich aus den Anforderungen eines realistischen Umfelds  ergeben. Der jährlich stattfindende Wettbewerb selbst ermöglicht es den Studenten, das eigene Können vor einer Jury aus Experten aus Wirtschaft und Wissenschaft zu präsentieren und sich mit anderen Hochschulteams zu messen.
%Aufgabe
%
%Das Studententeam wird von einem fiktiven Fahrzeughersteller beauftragt, anhand eines Modellfahrzeugs im Maßstab 1:10 ein möglichst kostengünstiges und energieeffizientes Gesamtkonzept eines autonomen Fahrzeuges zu entwickeln, herzustellen und zu demonstrieren. Beim Wettbewerb müssen möglichst schnell und fehlerfrei bestimmte Fahraufgaben bewältigt werden und das erarbeitete Konzept in Präsentationen erläutert werden.
%Bewertung
%
%Jedes Konzept wird mit den Konzepten der anderen teilnehmenden Teams verglichen und daraufhin bewertet. Hierzu müssen die Teams unterschiedliche statische und dynamische Disziplinen bestreiten, in denen man insgesamt 1000 Punkte erreichen kann.
%
%Statische Disziplinen
%Präsentation der Herstellungskosten und der Energiebilanz
%Präsentation des Einparkkonzeptes
%Präsentation des Spurführungskonzeptes mit Ausweichmanövern
%
%Dynamische Disziplinen
%Einparken parallel
%Rundstrecke ohne Hindernisse
%Rundstrecke mit Hindernissen
